\documentclass[11pt]{article}
\usepackage{fontspec,xunicode,polyglossia,longtable,csquotes,hyperref}
\setmainfont{Linux Libertine O}
\setmainlanguage{french}
\usepackage[a4paper]{geometry}
\begin{document}
\title{Statuts d'association suivant la loi de 1901}
\author{Association des Utilisateurs de \LaTeX\ en Sciences Humaines et Sociales}
\date{Automne~2013}
\maketitle


\section{Objet}

L'\emph{Association des Utilisateurs de \LaTeX\ en Sciences Humaines et Sociales} a pour objet :
\begin{itemize}
	\item De promouvoir l'utilisation du logiciel \LaTeX\ et des logiciels associés dans le domaine des Sciences Humaines et Sociales (SHS).
	\item De permettre la formation à ces logiciels pour les chercheurs des SHS.
	\item De réfléchir à l'amélioration de ces logiciels dans la perspective des besoins des SHS.
\end{itemize} 

\section{Moyens d'action}

L'association cherche à atteindre ses objectifs par le biais, non exclusifs, de :
\begin{itemize}
	\item La création et le maintien d'un site internet.
	\item Le maintien d'une liste électronique de discussion.
	\item L'organisation de rencontres.
	\item La publication d'articles et de livres sous forme numériques ou papiers.
\end{itemize} 

\section{Siège social et durée}

Le siège social est situé à l'Université de Picardie - Jules Verne. Il peut être transféré à la demande de l'assemblée générale extraordinaire.

La durée de l'association est illimitée.

\section{Adhésion}
Peuvent adhérer à l'association :
\begin{itemize}
	\item Les personnes physiques étudiants ou chercheurs dans le domaine des SHS.
	\item Les personnes physiques utilisateurs et dévellopeur de \LaTeX\ et les logiciels associés qui s'intéressent à son usage dans le domaine des SHS.
	\item Les unités de recherches du domaine des humanités numériques.
\end{itemize}

L'adhésion est de droit et gratuite pour les personnes physiques sus-mentionnés. 

Pour les unités de recherches, elle est conditionnée au soutien des activités de l'association, dont la forme peut-être :
\begin{itemize}
	\item Financement de rencontres ou de publications.
	\item Financement de développement de \LaTeX\ et logiciels connexes dans  la perspective des sciences humaines.
\end{itemize}

Le bureau statut sur une demande d'adhésion, dans un délai de deux mois après le dépôt.
Son renouvellement est automatique pour les personnes physiques. 
Pour les personnes morales, le bureau se prononce tout les deux ans sur son renouvellement.

Pour les personnes physiques, le décès ou la démission volontaire entraîne la perte de la qualité de membre.
Pour les personnes morales, la dissolution ou la démission volontaire entraîne la perte de la qualité de membre. 

\section{Ressources}

Les ressources de l'association se composent de :
\begin{itemize}
	\item Subventions.
	\item Dons d'adhérents.
	\item Droits d'auteur.
\end{itemize}

\section{Assemblée générale ordinaire}

L'Assemblée générale ordinaire se réunit à l'occasion des rencontres d'utilisateurs organisées par l'association. 
L'annonce de cette rencontre, au moins un mois avant, par le biais du site internet et de la liste de discussion, vaut convocation.

Si aucune AG n'a lieu dix-huit mois après la précédente AG, le bureau est tenu d'en organisé une dans les six mois.
Faut de quoi il est présumé démissionnaire et un nouveau bureau est tiré au sort, qui doit organiser une AG extraordinaire.

L'assemblée générale ordinaire :
\begin{itemize}
	\item Présente les bilans moral et financier du précédent bureau, qui sont ensuite votés.
	\item Élit un nouveau bureau.
\end{itemize}

\section{Assemblée générale extraordinaire}

L'assemblée générale extraordinaire peut être convoquée à la demande du bureau. Si trois membres de l'association la demande, le bureau est tenu de l'organiser dans les trois mois qui suivent.

L'organisation d'une AG extraordinaire n'est pas nécessairement corrélée à une rencontre d'utilisateurs. 
Une assemblée générale extraordinaire peut :
\begin{itemize}
	\item Modifier les statuts.
	\item Dissoudre le bureau et en élire un nouveau.
	\item Transférer le siège social.
	\item Dissoudre l'association.
\end{itemize}

Les votes s'y font à la majorité absolue des suffrages exprimés. Les scrutins sont publics, sauf lorsqu'ils concernent des personnes physiques ou morales.


\section{Bureau}

Le bureau est composé d'au moins trois membres et n'est pas limité en taille. Les membres sont élus au scrutin secret. Pour être élu, un candidat doit retenir la majorité absolue des suffrages exprimés.

Le bureau s'organise ensuite en interne, en désignant notamment :
\begin{itemize}
	\item Un(e) président(e).
	\item Un(e) secrétaire.
	\item Un(e) trésorier.
	\item Un(e) responsable technique chargés des outils internet.
\end{itemize}
Si le bureau ne contient que trois membres, l'un d'entre a une double fonction. 

Il a pour charge de :
\begin{itemize}
	\item Animer le site internet.
	\item Organiser les publications.
	\item Organiser les rencontres.
	\item Gérer les comptes.
	\item Proposer des modifications éventuelles des statuts.
\end{itemize}

Il peut demander à un ou plusieurs membres de l'association de l'aider dans ces tâches.
\end{document}
